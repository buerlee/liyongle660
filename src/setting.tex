\setmainfont{SourceSerifPro-Regular}
\usepackage{sourcesanspro}
\usepackage{sourcecodepro}
\setCJKmainfont{SourceHanSerifCN}%
[
	UprightFont = *-Regular,
	BoldFont = *-Bold,
	ItalicFont = 方正新楷体简体,
	BoldItalicFont = *-Bold,
	Mapping = fullwidth-stop
]
\setCJKsansfont{SourceHanSansCN}%
[
	UprightFont = *-Regular,
	BoldFont = *-Bold,
	ItalicFont = *-Regular,
	BoldItalicFont = *-Bold,
	Mapping = fullwidth-stop
]
\setCJKmonofont{SourceHanSansCN}%
[
	UprightFont = *-Regular,
	BoldFont = *-Bold,
	ItalicFont = *-Regular,
	BoldItalicFont = *-Bold,
	Mapping = fullwidth-stop
]
\usepackage{amsmath,amssymb,mwe}
\usepackage[upint]{newtxmath}
\usepackage{upgreek,physics,siunitx}% ,esint
\everymath{\displaystyle}
\edef\iint{\iint\limits}
\usepackage{lastpage}
\usepackage[paperwidth=15cm,paperheight=20cm,left=1cm,right=1cm,top=1.5cm,bottom=1.5cm]{geometry}
\setlength{\headheight}{13pt}

\usepackage{tikz,pgf}
\usetikzlibrary{shapes,calc}
\makeatletter
\newcommand\thumb{%
	\if@mainmatter
	\begingroup
	\catcode`\$=3
	\tikzpicture[remember picture,overlay] % thumb index
	\ifodd\value{page}
	  \node[fill=gray,text=black,anchor=north east,xshift=2mm,
			yshift=-10mm-\arabic{chapter}*20mm,
			shape=semicircle,shape border rotate=90,
			minimum height=10mm,minimum width=5mm,
			font=\normalfont\sffamily\bfseries\Huge]
		at (current page.north east)
		{\llap{\arabic{chapter}\hspace{1mm}}};
	\else
	  \node[fill=gray,text=black,anchor=north west,xshift=-2mm,
			yshift=-10mm-\arabic{chapter}*20mm,
			shape=semicircle,shape border rotate=270,
			minimum height=10mm,minimum width=5mm,
			font=\normalfont\sffamily\bfseries\Huge]
		at (current page.north west)
		{\rlap{\hspace{1mm}\arabic{chapter}}};
	\fi
	\endtikzpicture
	\endgroup
	\fi}
\makeatother

\usepackage{fancyhdr}
\pagestyle{fancy}
\fancyhf{}
\renewcommand{\footrulewidth}{0.4pt}
\fancyhead[RE]{\bfseries \leftmark}
\fancyhead[LO]{\bfseries \rightmark}
\fancyhead[RO,LE]{\thumb}
\fancyhead[C]{\bfseries 仅供学习使用,严禁商业使用}
\fancyfoot[C]{\bfseries --\thepage/\pageref{LastPage}--}
\fancypagestyle{plain}{%
\fancyhf{}
\fancyhead[r]{\thumb}
\fancyfoot[C]{\bfseries --\thepage/\pageref{LastPage}--}
\renewcommand{\headrulewidth}{0pt}
\renewcommand{\footrulewidth}{0pt}}

\usepackage{tcolorbox}
\tcbuselibrary{breakable,skins}
\newtcolorbox[auto counter]{ti}[1][]{%
enhanced,colback=white,colframe=white,attach boxed title to top center={yshift=-0.25mm-\tcboxedtitleheight/2,yshifttext=2mm-\tcboxedtitleheight/2},
boxed title style={
boxrule=0.5mm,frame code={ \path[tcb fill frame] ([xshift=-4mm]frame.west) -- (frame.north west) -- (frame.north east) -- ([xshift=4mm]frame.east) -- (frame.south east) -- (frame.south west) -- cycle; },
interior code={ \path[tcb fill interior] ([xshift=-2mm]interior.west) -- (interior.north west) -- (interior.north east) -- ([xshift=2mm]interior.east) -- (interior.south east) -- (interior.south west) -- cycle;}
},
fonttitle=\bfseries,breakable = false,title=\thetcbcounter,#1}
\usepackage{etoolbox}
% % part A
% \AtEndEnvironment{ti}{\tcblower%
% \parbox[t]{0.5\textwidth}{
% \begin{tabular}[c]{@{}|c|@{}}
% 	\hline
% 	答题\\
% 	区域\\
% 	\hline
% \end{tabular}}%
% \parbox[t]{0.5\textwidth}{%
% \begin{tabular}[c]{@{}|c|@{}}
% 	\hline
% 	纠错\\
% 	笔记\\
% 	\hline
% \end{tabular}
% }
% \\\vspace*{9em}
% }
% % part B
% \AtEndEnvironment{ti}{\tcblower%
% \parbox[t]{\textwidth}{
% \hspace*{\fill}
% \tikz \draw[dashed] (0,0) -- (0,-4);
% \hspace*{\fill}
% }}
\AtBeginEnvironment{tasks}{\kuo{}}
\usepackage{tasks}
\settasks{label=(\Alph*),label-width=15pt,after-item-skip=0pt}
\usepackage{hyperref}
\newcommand{\kuo}{\mbox{(\hspace{1.5em})}}
\newcommand{\hua}{\CJKunderline*[hidden=true]{瞻彼阕者,虚室生白}}
\newcommand{\ee}{\mathrm{e}}
\renewcommand{\leq}{\leqslant}
\renewcommand{\geq}{\geqslant}
\renewcommand{\partial}{\uppartial}
\title{2021 李永乐·王式安考研数学系列\\数学基础过关 660 题 数学一习题册\thanks{Build time:\today}}
\author{李永乐\thanks{清华大学} \and 王式安\thanks{北京理工大学} \and 武忠祥\thanks{西安交通大学}}
\date{2019 年 9 月}